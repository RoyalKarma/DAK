\clearpage
\section{Vysvětlete pojmy množství informace a entropie}
\begin{itemize}
    \item Množství informace = I
    \begin{itemize}
        \item Jde o množství informace, které nese  každá jednotlivá zpráva S$_i$
        \item Jednotkou je Shannon (Sh)
    \end{itemize}
    \item Entropie 
    \begin{itemize}
        \item Průměrné množství informace nesené jednou zprávou
    \end{itemize}
    \item Entropie Zdroje = H
    \begin{itemize}
        \item Průměrné množsví informace na jeden symbol
        \item Jednotka je Sh/symbol
    \end{itemize}
\end{itemize}

\section{ Vysvětlete rozdíl mezi jednotkami Shannon a bit}
\begin{itemize}
    \item Shannon je jednotka informace
    \item Bit je jednotka dat
    \item data != informace
\end{itemize}

\section{3.	Vysvětlete pojem základní zdroj zpráv, čím je určen a jaké jsou jeho vlastnosti}
Jedná se o diskrétní prvek, jelikož k popisu je oizžut jibečbž oičet charakteristických prvků.
Je určen množinou zpráv S = ${S_i}$´, kde každá s$_i$ je generována zdrojem s pravděpodobností p, přičemž platí, že $\sum p_i = 1$.
Množinu S nazýváme také abeceda zdroje, jednotlivé zprávy jsou symboly.

\section{Co je to nadbytečnost zdroje }
Nadbytečnost = R.
Jinak také \textit{redundance}.
R nabývá hodnot 0-1.
Vhodným kódováním a kompresí se snažíme snížit její hodnotu k 0.

\section{Vysvětlete pojem přizpůsobený zdroj zpráv a zdrojové kódovaní}
Používá se v praxi, kdy je potřeba přizpůsobit základní zdroj pro potřeby přenosu přenosovým kanálem.
Důvodem je nedostate k sigálových stavů vzhledem k počtu symbolů základního zdroje. Přízpůsobením zdroje na další přenost je použit pomocný zdroj, který má definován vyhovující počet symbolů. Pomocný zdroj je určen množinou X = ${x_i}$.
Zdrojové kódování jako přiřazení $s_i => x_i,x_s,x_k$\dots pro každý prvek abecedy X, hovoříme tedy o n-násobně rozšířeném zdroji informace. Zdrojové kódování se dále dělí na rovnoměrné (všechny značky stejné dlouhé, n=konst) a nerovnoměrné (značky jsou různých délek.

\section{Vysvětlete pojem účinnost kódování pro rovnoměrné a nerovnoměrné kódování}
Účinnost kódování je ekvivalentem entropie základního zdroje a je vyšší při použití nerovnoměrného kódování.

\section{7.	Jaké jsou statistické a dynamické charakteristiky přizpůsobeného zdroje zpráv}
Statistické
\begin{itemize}
    \item Entropie přizpůsobeného zdroje
    \item Max entropie přizpůsobeného zddroje
    \item Účinnost kódování
    \item Redundance přizpůsobeného zdroje
\end{itemize}
Dynamické
\begin{itemize}
    \item Informační rychlost přizpůsobého zdroje.
    \item Modulační rychlost.
    \item Přenosová rychlost prizpůsobeného zdroje.
    
\end{itemize}

\section{Co je to modulační rychlost? Jakou má jednotku?}
Modulační rychlost vyjadřuje počet změn prvků za jednotku času. Jednotkou je Bód [Bd].

\section{Uveďte popis chování systému pomocí vnějších a pomocí vnitřních stavových veličin}
Vnější
\begin{itemize}
    \item Popis systému matematickým modelem pomocí operátoru transformace T. Vztah mezi vstupy a výstupy lze poté zapsat pomocí rovnice $y=T(x)$.
    \item Využívá vnější veličiny
    \item Nejčastěji je použit k popisu lineárních dynamických  spojitých systémů s jednou vstupní a jednou výstupní veličinou.
    \item Podle jednoznačnosti operátoru T se systémy dělí na \textbf{deterministické} a \textbf{stochastické}.
    
\end{itemize}

Vnitřní
\begin{itemize}
    \item Používá se v případě, kdy metoda popisu pomocí vnějších proměnných není pro přehlednost použitelná.
    \item Metoda založena na pozorování vnitřních veličin tzv. \textit{stavu systému}
    \item Označení pro množinu definovaných podmínek, skutečnosti, nebo veličin, které můžeme v daném okamziku na systému rozpoznat => \textit{stavové proměnné}.
    \item Sledujeme-li v určitém časovém okamžiku \textit{t} hodnoty stavových , získáme \textit{stavový vektor} q(t).
\end{itemize}

\section{Vysvětlete rozdíl mezi deterministickým a stochastickým systémem}

Deterministický
\begin{itemize}
    \item Operátor \emph{T} jednoznačně převádí prvky vstupního vektoru na prvky výstupního vektoru.

\end{itemize}
Stochastický
\begin{itemize}
    \item Výskyt určitého prvku výstupního vektoru není jednoznačně určen.
    Každý vstupní prvek předchází v nějaký výstupní prvek z jisté množiny prvků s určitou pravděpodobností
    
\end{itemize}